\documentclass[12pt]{report}
\usepackage[margin=1in]{geometry}
\usepackage{fontspec}
\usepackage{xcolor}
\usepackage{authoraftertitle}
\usepackage{pythontex}
\usepackage{xltabular}
\usepackage{ragged2e}
\usepackage{parskip}

\title{Appendix}
\definecolor{msblue}{HTML}{5AB5D8}
\makeatletter

\begin{document}
\centerline{\textcolor{msblue}{
		\fontspec{Cambria}\textbf{\fontsize{13}{13}\MyTitle}
	}}


\bgroup\obeylines
\section*{First Meeting with Advisor}
\textbf{Me:} Hi dad, could you test out my app for a school project? It's an app that allows you to create a schedule for your day. To use it, you have to visit ib.timeblocker.io.
\textbf{Dad:} It's a website? Will there be a program I need to download?
\textbf{Me:} No, you won't need to download a program. Everything will be functional on the website alone.
\textbf{Dad:} Ok. (visits ib.timeblocker.io) What do I do now?
\textbf{Me:} You have to create an account using the Register button.
\textbf{Dad:} (visits the Register page) Do I have to use a valid email?
\textbf{Me:} No, it can be anything you want. So, what do you think of the pages so far? Do they look nice?
\textbf{Dad:} Yes, I think they look ok. (creates an account, gets redirected to the timeblock calendar page)
\textbf{Dad:} So I press the green button to create a timeblock?
\textbf{Me:} Yes.
\textbf{Dad:} Ok. (clicks the "add task" button). In the morning, I get up (creates a task called "get up"). I think, at 10AM (drags the "get up" task from the task sidebar and resizes it to fit between 10AM and 10:15AM). Then, I will need to go to the bank (creates a task called "go to bank" and drags it to fit between 11AM and 11:15AM). Then, I will need to go shopping (creates a task called "go shopping").
\textbf{Me:} What do you think of the interface?
\textbf{Dad:} I think it looks nice. But, I think the task font should be a bit larger.
\textbf{Me:} Ok.
\textbf{Dad:} So how do I create a schedule for a different day?
\textbf{Me:} You have to click the left arrow in the top left of the screen.
\textbf{Dad:} (returns to previous page) I also have to visit the Doctor on Monday (creates a timeblock for next Monday). Huh, why are the tasks I created still showing up?
\textbf{Me:} That's because you might still want to reuse those tasks.
\textbf{Dad:} But it looks messy and it's a bit confusing.
\textbf{Me:} I think that I will add a search bar that allows you to filter those tasks to only show some of them and to find certain tasks more easily.
\textbf{Dad:} (creates a task called "go to doctor" and resizes it between 1PM and 2PM) Alright, I think those are all the tasks I have.
\textbf{Me:} What did you think overall about the app?
\textbf{Dad:} I think it was good, I think adding a search bar and making the font size bigger would make the app better.
\textbf{Me:} Alright, thank you for testing it!

\section*{Second Meeting with Advisor}
\textbf{Me:} Hi dad, I've implemented the changes you wanted from our first meeting. Could you test the app again?
\textbf{Dad:} Is it still at the same website?
\textbf{Me:} Yes, you just need to visit ib.timeblocker.io
\textbf{Dad:} (visits the website and types in his credentials) Oh, the schedules saved from last time!
\textbf{Me:} Oh yeah, I forgot to tell you about that the first time.
\textbf{Dad:} (clicks on one of his timeblocks). I see the bigger font size, it looks much better.
\textbf{Me:} Thanks.
\textbf{Dad:} (types in "bank" in the search bar). And the search bar works.
\textbf{Me:} I also added a checkbox that allows you to sort the tasks by name.
\textbf{Dad:} (selects the checkbox and then unselects the checkbox) That's cool.
\textbf{Me:} Oh, and I forgot to mention the first time, there's also the ability to create a new column by pressing the plus icon in the top right.
\textbf{Dad:} (presses the "add column" button) What is the point of a new column?
\textbf{Me:} Well, it's for when you want to change your schedule in case it ends up getting interrupted.
\textbf{Dad:} Why can't you just modify the original schedule?
\textbf{Me:} I guess, so that you can preserve history and also compare your new schedule with your old one.
\textbf{Dad:} Oh, ok. (navigates back) I think the app is good.
\textbf{Me:} Do you think it would benefit students?
\textbf{Dad:} Yes, for sure.
\textbf{Me:} Alright, thank you for testing it once again!
\egroup


\section*{Third-party libraries}
My program makes extensive use of third-party libraries:

% \begin{noindent}
\begin{pycode}
from tex_escape import tex_escape

packages = [
	(
		"@fullcalendar/core",
		"FullCalendar provides a highly customizable and easy-to-use calendar to display on my website. I use it on the timeblock listing page where the user can create/select the timeblock for a certain date.",
		"Frontend"
	),
	(
		"@mdi/js",
		"""Provides a set of icons ("mdi" stands for "Material Design Icons") that improve the user interface; I've used these icons throughout various parts of Timeblocker.""",
		"Frontend"
	),
	(
		"quasar",
		"The Quasar Framework provides many useful UI components that I've used throughout the website.",
		"Frontend"
	),
	(
		"@quasar/extras",
		"Includes utilities provided by the @quasar/extras framework (e.g., a confirmation dialog when a user tries to delete a timeblock).",
		"Frontend"
	),
	(
		"vite",
		"A frontend build tool that allows changes in my file to be instantly displayed in my browser without me needing to refresh the page (refreshing the page would destroy the page state).",
		"Frontend"
	),
	(
		"@quasar/vite-plugin",
		"A Quasar plugin that allows me to use Quasar within Vite.",
		"Frontend"
	),
	(
		"@vitejs/plugin-vue",
		"A Vite plugin that enables using Vue with Vite (the build tool I'm using).",
		"Frontend"
	),
	(
		"sass",
		"A CSS Framework (I don't use it in Timeblocker directly; the Quasar package depends on it).",
		"Frontend"
	),
	(
		"tailwindcss",
		"A utility-first CSS framework that provides many out-of-the-box classes that can be easily embedded inside HTML. For me, it makes styling HTML much easier.",
		"Frontend"
	),
	(
		"@tailwindcss/typography",
		"A typography plugin for TailwindCSS that provides CSS classes that come with preconfigured classes that make prose text look great.",
		"Frontend"
	),
	(
		"autoprefixer",
		"A package that automatically adds vendor prefixes to CSS classes for supporting older browsers. I don't use it directly in Timeblocker; TailwindCSS depends on it.",
		"Frontend"
	),
	(
		"postcss",
		"A tool for transforming CSS with JavaScript. I don't use it directly in Timeblocker; TailwindCSS depends on it.",
		"Frontend"
	),
	(
		"@trpc/client",
		"The client package for tRPC that allows me to easily create type-safe API endpoints for the server. The client package is the package that makes the requests to the server.",
		"Frontend"
	),
	(
		"@vueuse/core",
		"Provides many useful Vue composition functions (e.g., useMouse() that returns reactive variables for the mouse position).",
		"Frontend"
	),
	(
		"daisyui",
		"A TailwindCSS components library that comes with many useful components (e.g. buttons and textboxes)",
		"Frontend"
	),
	(
		"dayjs",
		"A date-manipulation library for JavaScript.",
		"Frontend"
	),
	(
		"nanoid-nice",
		"A wrapper around nanoid (a custom ID generator) that filters out obscene words accidentally generated from random IDs.",
		"Frontend & Backend"
	),
	(
		"pinia",
		"A type-safe state management library for Vue.",
		"Frontend"
	),
	(
		"simple-vue-icon",
		"A customizable Vue component that easily integrates with @mdi/js to display an icon on the page.",
		"Frontend"
	),
	(
		"tw-elements",
		"Another TailwindCSS component library that provides useful UI components.",
		"Frontend"
	),
	(
		"vue",
		"A progressive JavaScript framework for building user interfaces.",
		"Frontend"
	),
	(
		"vue-input-autowidth",
		"A vue directive that makes a textbox automatically resize to the width of its text",
		"Frontend"
	),
	(
		"vue-router",
		"A router library for Vue that enables me to create separate web pages within Vue.",
		"Frontend"
	),
	(
		"@types/node",
		"TypeScript definitions for Node.",
		"Frontend & Backend"
	),
	(
		"typescript",
		"A strongly-typed superset of JavaScript that eliminates many type errors (e.g., from typos, passing wrong variable types to functions, calling a class method that doesn't exist on a certain class.",
		"Frontend & Backend"
	),
	(
		"vue-tsc",
		"A TypeScript checker that supports type checking Vue files (which can have TypeScript inside them).",
		"Frontend"
	),
	(
		"desm",
		"A utility package for retrieving the file path of the current file.",
		"Frontend & Backend"
	),
	(
		"prisma",
		"Prisma is a TypeScript ORM (Object-relational mapping) for SQL that is strongly-typed. Compared to raw SQL, Prisma is more secure, integrates better with TypeScript, and also supports extra features like database migrations. `prisma` is the CLI package that handles non-runtime tasks (e.g. schema generation, database migration, etc.).",
		"Backend"
	),
	(
		"@prisma/client",
		"The Prisma package that is used for runtime-tasks, like connecting to the database, executing SQL queries, etc.",
		"Backend"
	),
	(
		"@trpc/server",
		"The tRPC package that is used on the server to define the server's API endpoints and routes. Also provides features like input validation to verify that the frontend is sending correctly-formatted data.",
		"Backend"
	),
	(
		"nodemon",
		"A build tool that automatically reloads my Node server when my TypeScript files change.",
		"Backend"
	),
	(
		"bcrypt",
		"A hashing library for creating secure one-way hashes of users' passwords.",
		"Backend"
	),
	("fastify","A web framework for Node.js", "Backend"),
	("fastify-cookie","A Fastify plugin that adds supports for cookies.", "Backend"),
	("fastify-cors","A Fastify plugin that adds support for CORS (Cross-Origin Resource Sharing).", "Backend"),
	("fastify-plugin","A Fastify tool that allows me to create custom plugins.", "Backend"),
	("got","A tool that allows me to make requests to other websites.", "Backend"),
	("onetime","A function that ensures a function only gets run once.", "Backend"),
	("zod","A validation library.", "Backend"),
	("execa","A function that allows running code on the system.", "Backend"),
	("lion-system","A custom library containing useful tools for my workflow.", "Backend"),
]

def create_libraries_table():
	table_string = ""
	for (package_name, package_description, package_usage) in packages:
		table_string += f"""
			\\\\\\hline
			{tex_escape(package_name)} &
			{tex_escape(package_description)} &
			{tex_escape(package_usage)}
		"""
	return table_string
\end{pycode}
% \end{noindent}

\def\arraystretch{1.5}
\noindent\begin{xltabular}{\textwidth}{|
		p{0.3\textwidth\RaggedRight}|
		X|
		p{0.12\textwidth\RaggedRight}|}
	\hline
	Name &
	Description &
	Usage
	\py{create_libraries_table()}
	\\\hline
\end{xltabular}
% TODO: Add initial consultation with client and/or advisor
\end{document}