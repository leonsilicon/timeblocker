\documentclass[12pt, notitlepage]{article}
\usepackage[margin=1in]{geometry}
\usepackage{xltabular}
\usepackage{xcolor}
\usepackage{fontspec}
\usepackage{ragged2e}
\usepackage{bashful}
\usepackage{shellesc}
\usepackage{authoraftertitle}
\usepackage{hyperref}
\usepackage{pythontex}

\setpythontexcontext{jobname=\jobname, section=\thesection}

\makeatletter
\newcommand{\pythontexassign}[2]{%
  \expandafter\pythontexassign@i\expandafter{#2}{#1}}
\newcommand{\pythontexassign@i}[2]{%
  \pyc{#2="#1"}}
\makeatother

% \begin{noindent}
\begin{pycode}
import subprocess
from tex_escape import tex_escape
from shlex import split
def word_count():
	result = subprocess.run(f"""texcount -sub=section {pytex.context.jobname}.tex | grep "Section" | sed -e 's/+.*//'""", shell=True, stdout=subprocess.PIPE)
	word_counts = [s.strip() for s in result.stdout.decode('utf-8').splitlines()]
	return f"Word count: {tex_escape(word_counts[int(pytex.context.section) - 1])} words"
\end{pycode}
% \end{noindent}

\title{Criterion A: Planning}
\definecolor{msblue}{HTML}{5AB5D8}

\begin{document}
\centerline{\textcolor{msblue}{
		\fontspec{Cambria}\textbf{\fontsize{13}{13}\MyTitle}
	}}

\noindent\textbf{Client:} Leon Si (myself)
\\
\textbf{Advisor:} Mark Si (my dad)

\section{Problem Description}
Many people, especially students like me, have many tasks they need to finish in a day. However, an unstructured daily plan can be inefficient: that's where Timeblocking comes in.

Timeblocking is a strategy where you dedicate chunks of time of your day to work on specific tasks—I use timeblocking every day to help keep my schedule under control.

While timeblocking is a powerful strategy, there's no online program focused on timeblocking. When speaking with my advisor (see Appendix), he mentioned that there are current solutions like Google Calendar, but having tried these solutions myself, they make applying the specific strategy of timeblocking difficult and time-consuming. These apps don't have features that are essential for timeblocking, like an easy way to create a copy of a schedule to restructure your day.

Thus, I want to create an app from scratch that allows me (and other students!) to easily create a timeblocking schedule.

Because I would benefit from an timeblocking app, I have decided that I will be the client of this app, and my dad will be my advisor, helping me determine whether my app is easy to use for new users. I chose my dad as my advisor since he generally has a hard time learning to use new technology, so he would be invaluable for making sure that I design an app that is intuitive for new users to use.

\bigskip
\py{word_count()}

\section{Proposed Solution}
My app, Timeblocker, will be a website that allows allows students to create an account and add tasks to a daily timeblocking schedule. I decided to make my app a website so that students can access their schedules from any device with a browser, making it very accessible, which is important for a scheduling app.

I will be creating this website in TypeScript, a strongly-typed superset of JavaScript. I prefer TypeScript because it helps me catch errors such as typos and passing the wrong types to functions at compile-time instead of at runtime.

In order to implement accounts, I need to have a backend for Timeblocker that will be interacting with a database. I have decided to use PostgreSQL for the database since it allows me to declare relations between data (e.g. an account has many timeblocks). I will be creating the backend in TypeScript: by building the frontend and backend in the same language, I'll be able to share code between the frontend and backend, enabling me to reuse code.

By using accounts, students will be able to synchronize their schedule across different devices, making it very convenient. However, implementing accounts also comes with security concerns, since I will need to make sure I store the users' sensitive data such as their timeblock tasks and account passwords securely.

To host the website and server, I will be using GitHub Pages and Heroku, both of which are free. I estimate that development will take around 2-3 months.

\bigskip
\py{word_count()}

\section{Success Criteria}
\begin{itemize}
	\item Creating, editing, and delete tasks
	\item Drag tasks onto the schedule to create task blocks
	\item Creating multiple timeblocks
	\item Searching through task blocks
	\item Resizing the task blocks that are on the schedule
	\item Changing the schedule view between day and week
	\item Registering and logging into accounts
	\item Server-side synchronization with a database to access schedules across devices
	\item Schedule autosaves every time changes are made
	\item Exporting the timeblock schedule as a file
\end{itemize}

\end{document}